\subsection*{Project D3T-DYN (Focus on Stochastic Dynamical Patterns)}

\subsubsection*{Modelling the flow}
In order to  build, optimize and deploy a viable D3T solution, one must build a model
(discrete or continuous) to simulate transporter flow in a network. There has been various attempts on this,
from both communication theory aspect \cite{zhang_communication_2011, arenas_communication_2001}; and network flow aspect
\cite{de_martino_minimal_2009,sole-ribalta_model_2016,tan_hybrid_2013}. 

Some of the open questions are:
\begin{itemize}
\item Good criteria to judge if a model is appropriate. 
\item Development of statistical tests (cross-validation, predictive analysis, inference) to fit the parameters to such models. 
\item Studying universal phenomena like phase transitions with such models.
\end{itemize}

\subsubsection*{Building strategies}
Once these models are in place, we must build strategies: e.g. 
\begin{itemize}
\item Sharing,
\item Spatial distribution of transporters.,
\item Allocation of other resources.
\end{itemize}

Most crucial question would be: how to {\bf evaluate} such strategies. One might be inclined to think that certain characteristics
should be hallmarks of a good strategies: e.g. 
\begin{itemize}
\item quick saturation w.r.t benefit vs cost,
\item absence of sharp response of discomfort w.r.t economic benefit etc.
\end{itemize}