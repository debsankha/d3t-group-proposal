\subsection*{Project D3T-NET (Focus on Interplay of Network Structure and Dynamics)}
In D3T, the \emph{dynamics} includes components like 
\begin{itemize}
\item Ride sharing strategy (most crucial),
\item Routing strategy,
\item Resource allocation strategy,
\item Capacity/speed of the transporter
\end{itemize}
etc. The \emph{network}, in the most usual sense, can be some variation of the underlying transport network. However, 
conceptually completely different networks like sharability network \cite{santi_quantifying_2014} has also been proposed. 

It is well known that dynamical components \cite{zhang_probability_2013,guan_efficient_2013,santi_quantifying_2014} of the system qualitatively influences \emph{phenomena} like
congestion \cite{hyytia_congestive_2010, de_martino_congestion_2009}, phase transitions \cite{barankai_effect_2012}  etc. 

The broad question of how these influences change with network topology, is likely rich in possibilities.
